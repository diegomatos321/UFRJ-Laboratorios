\documentclass[]{article}
\usepackage{listings}
\usepackage{graphicx}

%opening
\title{Trabalho Final Seg Info}
\author{Diego V. S. de Matos -- 120098723}

\begin{document}

\maketitle

\section{Protocolo TLS}\\
O TLS é um protocolo de segurança amplamente implantado acima da camada de transporte que é suportado por quase todos os navegadores, servidores web através da porta 443, fornece:
\begin{itemize}
	\item confidencialidade: via criptografia simétrica
	\item integridade: via hash criptográfico
	\item autenticação: via criptografia de chave pública
\end{itemize}

\section{Comparação de Implementações}
Foi implementado em Python dois programas cliente e servidor para cada caso: com TLS e sem TLS. As rotinas para cada caso são parecidas e suas diferenças principais são:

1) Inicialização dos programas: No caso com TLS o servidor deve carregar os certificado criar seu socket normalmente

\begin{lstlisting}[caption={Inicializacao Servidor}]
# Rotina abaixo apenas para o caso com TLS
context = ssl.SSLContext(ssl.PROTOCOL_TLS_SERVER)
context.load_cert_chain(certfile="cert.pem", keyfile="key.pem")

# Rotina abaixo em comum para ambos os casos
server = socket.socket(socket.AF_INET, socket.SOCK_STREAM)
server.bind((HOST, PORT))
server.listen(1)		
\end{lstlisting}

2) Envólucro na conexão: Na conexão com TLS é feita a encriptação/decriptação automática no conteúdo da mensagem tanto no cliente como no servidor. Passo não necessário no caso sem TLS.

\begin{lstlisting}[caption={Conexão servidor}]
conn, addr = server.accept()
# Conexao segura TLS
secure_conn = context.wrap_socket(conn, server_side=True)
\end{lstlisting}

\begin{lstlisting}[caption={Conexão cliente}]
# Apenas para o caso TLS
context = ssl.create_default_context()
context.check_hostname = False
context.verify_mode = ssl.CERT_NONE  # aceitar certificado autoassinado
	
# O caso sem TLS chama apenas a funcao abaixo
raw_sock = socket.socket(socket.AF_INET, socket.SOCK_STREAM)
	
# Envolucro na conexao socket do cliente
secure_sock = context.wrap_socket(raw_sock, server_hostname=HOST)
\end{lstlisting}

\section{Captura dos Pacotes}

\begin{figure}[htb]
	\centering
	\includegraphics[width=\linewidth]{./captura-sem-tls.png}
	\caption{Pacotes enviados sem tls}
	\label{fig:captura-sem-tls}
\end{figure}
Na figura~\ref{fig:captura-sem-tls} é possível compararmos o fluxo de pacotes para cada implementação. No caso sem TLS observamos uma sequência de pacotes padrão do protocolo TCP (SYN, ACK) entre o cliente (porta 38898) e o servidor (porta 5000) além de conseguirmos ler o conteúdo da mensagem como destacado na imagem. %descrever a ordem das pacotes

\begin{figure}[htb]
	\centering
	\includegraphics[width=\linewidth]{./captura-com-tls.png}
	\caption{Pacotes enviados com tls}
	\label{fig:captura-com-tls}
\end{figure}
Já para o caso com TLS observamos na figura~\ref{fig:captura-com-tls} os pacotes o TLS e não conseguimos ler o conteúdo da mensagem.

\end{document}
